\documentclass{article}

\title{School of Block Whitepaper}
\date{2018-10-01}
\author{Gwendolyn Faraday}

\begin{document}
  \pagenumbering{gobble}
  \maketitle
  \newpage
  \pagenumbering{arabic}

  \section{The Problem}

  There are a plethora of high quality, free learning materials all over the internet. However, these resources are not connected in meaningful ways. Someone desiring to learn a subject must search do a lot of searching and researching before beginning to learn, unless they go to an expensive school or specialized traning program. Thus, there is unnecessary time and frustration spent when trying to scaffold self-taught curriculums to learn with.

  \section{The Solution}

  School of Block aims to solve these problem by incentivising users to learn, currate, and create content. The platform will be free for learners and other participants - curators, content creators, etc. School of Block will also have tokenized rewards to incentivize each type of user.

  \subsection{Product Architecture}

  Users can interact with the application, pass challenges and work through lessons. The lessons will be grouped into blocks to learn a specific subject. These blocks can be grouped into customizable learning plans.
  
  Teachers will be able to create learning plans based off of their own content or other online learning materials. Anyone can view and use these learning plans.
  
  [image]

  \subsection{First Users}

  Our initial market will be self-learners who are interested in any aspect of blockchain development. We will also appeal to educators interested in curating or creating content.
  
  In the future, the content will expand beyond blockchain to encompass other types of technology education.

  \subsection{Development Strategy}

  The initial site build will be a prototype web app. Next, we will turn that into a hybrid application and introduce a reward token.

  \section{School Token Overview}

  The main token will initially be an ERC20 standard token to provide rewards within the ecosystem. There will also be non-fungible tokens (ERC721) used as incentives for participating in the community later on.

  \section{The Team}

  Gwendolyn Faraday (co-founder)
  Akshay Chaudhary (co-founder)

  \section{Plan for Issuing Tokens}

  We are still deciding on the ratio of tokens to allot for different things. Some tokens will go to developing the application, some will be set aside for rewards in the ecosystem, and some may be used for investors or for development bounties.

\end{document}

